\documentclass[12pt, letterpaper]{article}
\usepackage{amsmath}
\usepackage{amssymb}
\usepackage{mathtools}

\usepackage{physics}
\usepackage{siunitx}
\usepackage{cancel}

\usepackage{bm}
\usepackage{mathrsfs}
\usepackage{xfrac}

\usepackage{array}
\usepackage{booktabs}
\usepackage{geometry}
\usepackage{multicol}

\title{My Own CRM}
\author{Par Moi}
\date{Novembre 2025}
\begin{document}
\maketitle
\noindent Ce document est écrit en \LaTeX{}
\\

\hspace*{2em}\textbf{\underline{Mathématique}}
\newline
\newline
\textbf{\underline{Equation différentielle:}}
\\
\\
\textbf{Equations linéaire du premier ordre :}
\\
Nous allons premièrement résoudre les équations linéaire du premier ordre, donc de la forme : $y' + a(x)y = b(x)$
\newline
Nous commençons par définir : $h(x) = e^{\int a(x)\, dx}$
\newline
\[
y' + a(x)y = b(x) 
\]
\[
\Leftrightarrow h(x)y' + h(x)a(x)y = h(x)b(x)
\]
\[
\Leftrightarrow \frac{d}{dx} \biggl( h(x) y \biggr) = b(x) h(x)
\]
\[
\Leftrightarrow y = \frac{\int b(x)h(x)\, dx + c}{h(x)}
\]
\newline
Notre solution générale s'écrit donc :
\[
y(x) = \left(\int b(x)e^{\int a(x)\, dx}\, dx + c \right)e^{-\int a(x)\, dx}
\]
\newpage
\noindent\textbf{Equation différentielle linéaire du second degré à coefficient constants :}
\\
\\
Nous commençons par nous intéresser au équations homogènes : $ay'' + by' + cy = 0$ où $a,b,c$ sont des coefficients constants.
\\
Nous cherchons une solution de la forme : $f(x) = e^{\lambda x}$
\\
$f(x)$ est solution de l'équation si et seulement si $\lambda$ est solution de : $a\lambda^2 +b\lambda + c$
\\
\\
Cas réel :
\\
Il s'agit du cas où $a,b,c \in \mathbb{R}$ avec $a \neq 0$
\\
\\
Si $\Delta > 0$ : \\
L'équation possède deux solutions réelles, données par $\lambda_{1,2} = \frac{-b \pm \sqrt{\Delta}}{2a}$.
\\
Les solutions sont donc définies par $f(x) = c_1e^{\lambda_1 x} + c_2e^{\lambda_2 x}$.
\\
\\
Si $\Delta = 0$ : \\
L'équation possède une solution réelle, donnée par $\lambda = \frac{-b}{2a}$
\\
Les solutions sont donc définies par $f(x) = (c_1x + c_2)e^{\lambda x}$.
\\
\\
Si $\Delta < 0$ : \\
L'équation possède deux solutions complexes: $a + ib$ et $a-ib$.
\\
Les solutions sont donc définies par $f(x) = c_1e^{ax}sin(\abs{b}x + c_2)$
\\
\\
Cas Complexe :
\\
Il s'agit du cas où $a,b,c \in \mathbb{C}$ avec $a \neq 0$
\\
\\
Si $\Delta \neq 0$ :
\\
L'équation possède deux solutions distinctes: $\lambda_1$ et $\lambda_2$
\\
Les solutions sont donc définies par $f(x) = C_1e^{\lambda_1 x} + C_2e^{\lambda_2 x}$.
\\
\\
Si $\Delta = 0$ : \\
L'équation possède une solution, donnée par $\lambda = \frac{-b}{2a}$
\\
Les solutions sont donc définies par $f(x) = (C_1x + C_2)e^{\lambda x}$.
\newpage
\noindent \textbf{Equation différentielle linéaire du second degré à coefficient variables :}
\\
\\
Nous nous intéressons aux équations de la forme : $a(x)y'' + b(x)y' + c(x)y = 0$ 
\\
Il n'existe pas d'expression générale pour determiner des solutions d'une telle équation. \\
Toutefois, il est possible de la résoudre complètement si une solution paticulière non nulle de l'équation est connue.
\\
Prenons notre solution particulierre que nous noterons $f(x)$.
\\
Le problème revient à résoudre une équation différentielle linéaire d'ordre 1 :
\\
\[
f(x)y_1' - f'(x)y_1 = c_0 e^{\int -\frac{b(x)}{a(x)} dx} 
\]
\[
\Leftrightarrow y_1' - \frac{f'(x)}{f(x)}y_1 = \frac{c_0 e^{\int -\frac{b(x)}{a(x)} dx} }{f(x)}
\]
\[
\Leftrightarrow \frac{d}{dx} \left( y_1e^{-\int \frac{f'(x)}{f(x)} dx} \right) = \frac{c_0 e^{\int -\frac{b(x)}{a(x)} dx} e^{-\int \frac{f'(x)}{f(x)} dx}}{f(x)}
\]
\[
\Leftrightarrow \frac{d}{dx} \left( \frac{y_1}{f(x)} \right) = \frac{c_0 e^{\int -\frac{b(x)}{a(x)} dx}}{f^2(x)}
\]
\[
\Leftrightarrow \frac{y_1}{f(x)} = \int \frac{c_0 e^{\int -\frac{b(x)}{a(x)} dx}}{f^2(x)} dx +c
\]
\[
\Leftrightarrow y_1 = f(x) \left(\int \frac{c_0 e^{\int -\frac{b(x)}{a(x)} dx}}{f^2(x)} dx +c\right)
\]
Notre fonction $y_1$ est la solution générale de l'équation différentielle homogène.
\newpage
\noindent \textbf{Equation différentielle linéaire du second degré non-homogène :}
\\
Nous nous intéressons aux équations de la forme : $a(x)y'' + b(x)y' + c(x)y = d(x)$
\\
On se servira des solutions des équations différentielles homogènes vues précedemment.
\\
Une solution générale à notre équation peut être donnée grâce à l'addition d'une solution particulière et de la solution générale de l'équation homogène associée.
\\
\\
Si $d(x)$ est la somme de plusieurs fonctions, nous pouvons chercher une solution particulière pour chacune de ces fonction et les additionné pour avoir notre solution particulière.
\\
\\
Recherche d'une solution particulière :
\\
On cherche une solution sous la forme $y_p(x) = u_1(x)y_1(x) + u_2(x)y_2(x)$ avec $y_1$ et $y_2$ solution indépendante de l'équation homogène et $u_1$ et $u_2$ à trouver.
\\
Pour faciliter la tâche, on impose : $u_1'y_1 + u_2'y_2 = 0$
\\
On a donc :
\[
y_p' = u_1y_1' +u_2y_2'
\]
Et :
\[
y_p'' = u_1y_1'' +u_2y_2'' + u_1'y_1' +u_2'y_2'
\]
On peut placer $y_p$ et ses dérivés dans l'équation d'orgine ce qui nous donne après simplification:
\[
u_1'y_1' + u_2'y_2' = d(x)
\]
Nous avons donc le système d'équation:
\[
\begin{cases}
  u_1'y_1' + u_2'y_2' = d(x) \\
  u_1'y_1 + u_2'y_2 = 0
\end{cases}
\]
Ce qui nous permet de trouver $u_1$ et $u_2$ :
\[
u_1 = \int \frac{d(x)}{y_1' - \frac{y_1y_2'}{y_2}}dx \hspace*{2em} u_2 = \int \frac{d(x)}{y_2' - \frac{y_2y_1'}{y_1}}dx 
\]
On peut donc trouver $y_p(x) = u_1(x)y_1(x) + u_2(x)y_2(x)$
\\
La solution générale nous est donnée par : $y_g = y_p + y_h$ où $y_h$ est la solution générale de l'équation homogène.
%
%
\newpage
\noindent \textbf{\underline{Equations différentielles de Bernoulli :}}
\\
\\
Nous nous intéressons aux équations de la forme : $y' + a(x)y = b(x)y^m$
\[
y' + a(x)y = b(x)y^m
\]
\[
\Leftrightarrow y'y^{-m} + a(x)y^{1-m} = b(x)
\]
Nous définissons $t(x) = y^{1-m}$, donc $t'(x) = (1-m)y^{-m}y'$
\[
\frac{t'}{1-m} + a(x)t = b(x)
\]
\[
\Leftrightarrow t' + (1-m)a(x)t = b(x)(1-m)
\]
\[
\Leftrightarrow \frac{d}{dx}\left(te^{(1-m) \int a(x) dx}\right)= b(x)(1-m)e^{(1-m) \int a(x) dx}
\]
\[
\Leftrightarrow te^{(1-m) \int a(x) dx}= \int b(x)(1-m)e^{(1-m) \int a(x) dx} dx + c
\]
\[
\Leftrightarrow t= \left(\int b(x)(1-m)e^{(1-m) \int a(x) dx} dx + c\right) e^{(m-1) \int a(x) dx}= y^{1-m}
\]
\[
\Leftrightarrow y= \sqrt[1-m]{\left(\int b(x)(1-m)e^{(1-m) \int a(x) dx} dx + c\right) e^{(m-1) \int a(x) dx}}
\]
\newpage
\noindent \textbf{\underline{Algèbre :}}
\newline
\newline
\underline{Identité :}
    \begin{center}
        \begin{tabular}{|c|c|}
        \hline
        \rule{0pt}{1.2em}
        $(a+b)^2 = a^2+2ab+b^2$ & $(a-b)^2 = a^2-2ab+b^2$ \\[2pt]
        \hline
        \rule{0pt}{1.2em}
        $(a+b)^3 = a^3+3a^2b+ 3ab^2+b^3$ & $(a-b)^3 = a^3-3a^2b+ 3ab^2-b^3$ \\[2pt]
        \hline
        \multicolumn{2}{|c|}{\rule{0pt}{1.2em}$(a+b)^n = \sum_{k=0}^{n}\binom{n}{k}a^{n-k} b^k$} \\[2pt]  % prend les 2 colonnes
        \hline
        \rule{0pt}{1.2em}
        $a^2-b^2 = (a-b)(a+b)$ & $a^2+b^2 = (a-ib)(a+ib)$\\[2pt]
        \hline
        \end{tabular}
    \end{center}
\underline{Puissances et racines :}
\\
Nous notons $a$ et $b$ des nombres strictements positifs; $\sqrt[n]{a}$ n'est définie que pour $n \in \mathbb{N}$
    \begin{center}
        \begin{tabular}{|l|l|l|l|l|}
        \hline
        \rule{0pt}{1.8em}
        $a^0 = 1$ & $a^p = a\cdot a^{p-1}$ & $a^{-q} = \frac{1}{a^q}$ & $a^{\frac{1}{q}} = \sqrt[q]{a}$ & $a^{\frac{p}{q}} = \sqrt[q]{a^p}$\\[8pt]
        \hline
        \rule{0pt}{1.4em}
        $a^pa^q = a^{q+p}$ & $\frac{a^p}{a^q} = a^{p-q}$ & $\left(a^p\right)^q = a^{pq}$ & $a^pb^p = (ab)^p$ & $\frac{a^p}{b^p} = \left(\frac{a}{b}\right)^p$\\[4pt]
        \hline
        \rule{0pt}{1.4em}
        $\left(\sqrt[n]{a}\right)^n = a$ & $\left(\sqrt[q]{a}\right)^p = \left(\sqrt[q]{a^p}\right)$ & $\sqrt[p]{\sqrt[q]{a}} = \sqrt[pq]{a}$ & $\sqrt[p]{a}\sqrt[p]{b} = \sqrt[p]{ab}$ & $\sqrt[p]{\frac{a}{b}} = \frac{\sqrt[p]{a}}{\sqrt[p]{b}}$\\[4pt]
        \hline
        \end{tabular}
    \end{center}
\underline{Logarithmes :}
\\
Nous notons $a$ un nombre réel strictement positif et différent de 1.
\\
$y = \log_{a}(x) \Leftrightarrow a^y = x$ \hspace*{2em} $y$ est le \textit{logarithme en base} $a$ de $x$, pour $x \in \mathbb{R}_+^*$ 
    \begin{center}
        \begin{tabular}{|l|l|}
        \hline
        \rule{0pt}{1.8em}
        $\log_a(xy) = \log_a(x) + \log_a(y)$ & $\log_a\left(\frac{x}{y}\right) = \log_a(x)-\log_a(y)$ \\[8pt]
        \hline
        \rule{0pt}{1.4em}
        $\log_a\left(\frac{1}{y}\right) = -\log_a(y)$ & $\log_a(x^p) = p \cdot \log_a(x)$ \\[8pt]
        \hline
        \end{tabular}
    \end{center}

\newpage
\underline{Nombres complexes :}
\\
\\
Nous notons $i$, un nombre tel que $i^2 = -1$
\\
Forme algébrique : $z = a+ib$ où $a,b \in \mathbb{R}$
\\
\hspace*{2em}$a$ est la \textit{partie réelle} de $z$, notée $Re(z)$
\\
\hspace*{2em}$b$ est la \textit{partie imaginaire} de $z$, notée $Im(z)$
\\
Forme trigonométrique : $z = r (cos(\theta)+i sin(\theta)) = r cis(\theta)$ avec $r \in \mathbb{R}_+$ et $\theta \in \mathbb{R}$
\\
\hspace*{2em}$r$ est le \textit{module} de $z$, noté $\abs{z}$
\\
\hspace*{2em}$\theta$ est l'\textit{argument} de $z$, noté $arg(z)$
\\
Forme exponentielle : $z = re^{i\theta}$
\\
\\
\textbf{Relations entre les différentes formes de $z$ :}
    \begin{center}
        \begin{tabular}{|c|c|}
        \hline
        \rule{0pt}{1.2em}
        $r = \sqrt{a^2+b^2}$ & $tan(\theta) = \frac{b}{a}$ \\[2pt]
        \hline
        \rule{0pt}{1.2em}
        $cos(\theta) = \frac{a}{\sqrt{a^2+b^2}}$ & $sin(\theta) = \frac{b}{\sqrt{a^2+b^2}}$ \\[2pt]
        \hline
        \rule{0pt}{1.2em}
        $a = rcos(\theta)$ & $b = rcos(\theta)$\\[2pt]
        \hline
        \multicolumn{2}{|c|}{\rule{0pt}{1.2em}Formule d'Euler : $e^{i\theta} = cos(\theta) + i sin(\theta)$} \\  % prend les 2 colonnes
        \hline
        \end{tabular}
    \end{center}
\textbf{Opérations sur les nombres complexes :}
    \begin{center}
        \begin{tabular}{|l|l|}
        \hline
        Forme algébrique : & Forme exponentielle : \\
        \hline
        $z_1 + z_2 = (a_1 + a_2) + (b_1+b_2)i$ &  \\
        \hline
        \rule{0pt}{1.2em}
        $z_1z_2 = (a_1a_2-b_1b_2) + (a_1b_2+a_2b_1)i$ & $z_1z_2 = r_1r_2e^{i(\theta_1 + \theta_2)}$ \\[2pt]
        \hline
        \rule{0pt}{1.4em}
        $\frac{z_1}{z_2} = \frac{a_1a_2+b_1b_2}{a_2^2+b_2^2} + \frac{a_2b_1+a_1b_2}{a_2^2+b_2^2}i$ & $\frac{z_1}{z_2} = \frac{r_1}{r_2}e^{i(\theta_1-\theta_2)}$\\[2pt]
        \hline
        \rule{0pt}{1.4em}
        $\frac{1}{z} = \frac{a}{a^2+b^2} - \frac{b}{a^2+b^2}i$ & $\frac{1}{z} = \frac{1}{r_1}e^{-i\theta}$\\[2pt]
        \hline
        \rule{0pt}{1.4em}
        & $z^n = r^ne^{in\theta}$\\[2pt]
        \hline
        \end{tabular}
    \end{center}
\textbf{Formule de Moivre :}
\\
$(cos(\theta) + isin(\theta))^n = cos(n\theta) + isin(n\theta)$
\\
\\
\textbf{Racine \textit{n}-ièmes}
\\
On note $z = re^{i\theta}$, un nombre complexe non nul.
L'équation $w^n = z, n \in \mathbb{N}$, possède n solutions distinctes :
\[
w_k = \sqrt[n]{r}\hspace*{0.2em}e^{i\frac{\theta+2k\pi}{n}}\hspace*{2em} k = 0,1,2,...,n-1
\]
\\
\textbf{Conjugué}
\\
Le \textit{conjugué} de z est $\bar{z} = a-bi = re^{-i\theta}$
    \begin{center}
        \begin{tabular}{|l|l|l|l|}
        \hline
        \rule{0pt}{1.8em}
        $\overline{z_1+z_2} = \overline{z_1} + \overline{z_2}$ & $\overline{z_1z_2} = \overline{z_1}\hspace*{0.2em}\overline{z_2}$ & $\overline{\left(\frac{z_1}{z_2}\right)} = \frac{\overline{z_1}}{\overline{z_2}}$ & $z\overline{z} = \abs{z}^2$\\[8pt]
        \hline
        \rule{0pt}{1.4em}
        $Re(z) = \frac{1}{2}(z + \overline{z})$ & $Im(z) = \frac{1}{2i}(z-\overline{z})$ & $\overline{\overline{z}} = z$ & $\frac{1}{z} = \frac{\overline{z}}{\abs{z}^2}$\\[4pt]
        \hline
        \end{tabular}
    \end{center}
\newpage
\textbf{Source :} \\
%https://fr.wikipedia.org/wiki/%C3%89quation_diff%C3%A9rentielle_lin%C3%A9aire_d%27ordre_deux#%C3%89quation_diff%C3%A9rentielle_homog%C3%A8ne
%this is to write comment
\end{document}