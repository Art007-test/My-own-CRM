\documentclass[12pt, letterpaper]{article}
\usepackage{amsmath}
\usepackage{amssymb}
\usepackage{mathtools}

\usepackage{physics}
\usepackage{siunitx}
\usepackage{cancel}

\usepackage{bm}
\usepackage{mathrsfs}
\usepackage{xfrac}

\usepackage{array}
\usepackage{booktabs}
\usepackage{geometry}
\usepackage{multicol}

\title{My Own CRM}
\author{Par Moi}
\date{Novembre 2025}
\begin{document}
\maketitle
\noindent Ce document est écrit en \LaTeX{}
\\

\hspace*{2em}\textbf{\underline{Mathématique}}
\newline
\newline
\textbf{\underline{Equation différentielle:}}
\\
Nous allons premièrement résoudre les équations linéaire du premier ordre, donc de la forme: $y' + a(x)y = b(x)$
\newline
Nous commençons par définir $h(x) = e^{\int a(x)\, dx}$
\newline
\[
y' + a(x)y = b(x) 
\]
\[
\Leftrightarrow h(x)y' + h(x)a(x)y = h(x)b(x)
\]
\[
\Leftrightarrow \frac{d}{dx} \biggl( h(x) y \biggr) = b(x) h(x)
\]
\[
\Leftrightarrow y = \frac{\int b(x)h(x)\, dx + c}{h(x)}
\]
\newline
Notre solution générale s'écrit donc:
\[
y(x) = \left(\int b(x)e^{\int a(x)\, dx}\, dx + c \right)e^{-\int a(x)\, dx}
\]
Equation différentielle linéaire du second degré à coefficient constants:\\
Equation différentielle linéaire du second degré à coefficient variables:
\newpage
\noindent \textbf{\underline{Algèbre:}}
\newline
\newline
\underline{Identité:}
\begin{center}
\begin{tabular}{|c|c|}
\hline
$(a+b)^2 = a^2+2ab+b^2$ & $(a-b)^2 = a^2-2ab+b^2$ \\
\hline
$(a+b)^3 = a^3+3a^2b+ 3ab^2+b^3$ & $(a-b)^3 = a^3-3a^2b+ 3ab^2-b^3$ \\
\hline
\multicolumn{2}{|c|}{$(a+b)^n = \sum_{k=0}^{n}\binom{n}{k}a^{n-k} b^k$} \\  % prend les 2 colonnes
\hline
$a^2-b^2 = (a-b)(a+b)$ & $a^2+b^2 = (a-ib)(a+ib)$\\
\hline
\end{tabular}
\end{center}
\underline{Puissances et racines:}
\\
\\
\underline{Nombres complexes:}
\\
Nous notons i, un nombre tel que $i^2 = -1$
\\
Forme algébrique: $z = a+ib$ où $a,b \in \mathbb{R}$
%this is to write comment
\end{document}